\section{Conclusion}
% \andrew{Lead with the positive, then follow up with the equivocal / negative. So put the lack of significant efficiency gains as the second sentence, after the original sentence about faster editing.}
% Our lab study showed that FFL led to faster and easier editing of augmentation markup compared to a LaTeX baseline, while yielding more readable markup\zed{, on complex editing tasks. On simpler tasks involving creating augmentations from scratch, there was no significant improvement}.
\zed{Our controlled lab study yielded two results. First, in complex editing tasks, FFL led to faster and easier editing of augmentation markup compared to a LaTeX baseline, while yielding more readable markup. Second, for simpler tasks where authors wrote simple augmentations from scratch, we observed no significant differences between the two.}
Our study offers signs that FFL reduces viscosity, hard mental operations, and error proneness, while supporting closeness of mapping and and progressive evaluation. This paper demonstrates the potential of tools like FFL that extend authoring environments to support the practice of augmenting notation. We hope tools like FFL bring about more pervasive authoring of approachable explanations of math notation.
% Participants generally acknowledged the value of many core FFL features including instant feedback, the separation between annotation code and formula code, and the enhanced capability to make cross-cutting changes, all of which can be considered in design decisions for maths augmentation tools. With much desire for more features and refinements, FFL can potentially grow into a useful authoring tool that simplifies the styling of math expressions embedded into multiple scenarios including text editors, slide editors
% and video tutorial editors for broader purposes.

% Despite much desire for more features and refinements, preliminary results do
% show that FFL could grow into a useful tool that simplifies styling of math
% expressions. As we continue to refine the implementation from our minimum viable
% product, we see many of these features within reach, even if not immediately approached.
% We will also conduct some basic formal user studies to inform said refinements
% in our design in the same process.