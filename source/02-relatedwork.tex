\section{Background and Related Work}

% SHORTER VERSION:
In this section, we discuss why authors might wish to augment notation and then situate our system amid related work.
% In this section, we discuss why authors might wish to augment notation and then situate our work relative to prior tools for authoring and augmenting formulas.

\subsection{Notation and augmentation}

Math notation is difficult to read. It has been described as a language of its own, requiring practice to understand~\cite{ref:adams2003reading}. Over time, the human perceptual system can become trained to recognize structures in formulas~\cite{ref:marghetis2016mastering}. Readers learn idioms in formulas through repeated exposure, such that experts can spot structures in formulas novices miss~\cite{ref:shepherd2014reading}. For novices, notation poses a barrier to understanding mathematical texts and is often cited as a challenge in self-teaching machine learning~\cite{ref:cai2019software} and reading research papers~\cite{ref:mysore2023how}.

Subtle changes to the presentation of notation can affect its readability. For instance, coloring and annotating formulas can reduce cognitive load in solving algebra problems~\cite{ref:yung2015effects}. Readers can be aided in understanding operator precedence by altering which letters are used for variables and spacing between variables~\cite{ref:goldstone2017adapting,ref:harrison2020spacing}. The design space for augmented notation is large. \citet{ref:dragunov2003designing} propose augmenting formulas with symbols definitions, annotations that show how variables are manipulated across stages of derivation, and controls that adjust the level of detail in a derivation. \citet{ref:head2022math} and Hohman et al.~\cite{ref:hohman2020awesome} expand this design space, with the former describing 16 classes of augmentations.
In this paper, we explore how authoring environments could be extended to equip authors with tools to perform common some of the most common kinds of augmentations.

\subsection{Tools for augmenting notation}

% Authors have at their disposal many tools for augmenting notation, including markup languages for typesetting formulas, graphical editors, and sketching tools. A recent interview study by \citet{ref:head2022math} suggests that authors using these tools encounter friction of various kinds. This paper is particularly focused on improving the experience of reading and writing markup code, experimenting with cross-document styling rules, and baking good visual design into augmentations.

\paragraph{Markup languages}

One of the most common kinds of tools for writing and augmenting notation is the markup language. Markup languages, like TeX~\cite{ref:knuth1986tex}, allow authors to write formulas in plain text and render them as cleanly typeset formulas. Some such tools provide support for augmentation. LaTeX~\cite{tool:latex}, for instance, supports the addition of color with the \texttt{color}~\cite{tool:carlisle1997color} package, and labels with macros from the \texttt{mathtools}~\cite{tool:madsen2014mathtools} and \texttt{annotate-equations} ~\cite{tool:annotateequations} packages. Recent research suggests that these tools could benefit from cleaner markup design, better defaults, and better support for cross-document style changes~\cite{ref:head2022math}.

The popularity of TeX as a language for formula typesetting has led to web-based TeX formula typesetters. One such tool is KaTeX~\cite{tool:katex}. The context of the web provides new opportunities for augmentation. KaTeX offers authors the \texttt{\textbackslash{}htmlClass} for assigning HTML classes to arbitrary expressions. CSS can then be used to apply styles to those expressions. Our goal with FFL was to support augmentation of TeX formulas in web-based authoring environments using a language similar to CSS.

Notation augmentation is a feature of several recent markup languages for math and science communication. Nota~\cite{ref:crichton2021new} and Heartdown~\cite{ref:li2022heartdown} let authors specify definitions of symbols, revealing symbol definitions in the margins of selected formulas when clicked upon. Curvenote's editor API~\cite{tool:curvenote} provides support for parametric LaTeX formulas, where numeric values can be substituted into formulas as users interact with widgets. \emph{manim}~\cite{tool:manim} supports the creation of animations of math formulas with step-by-step builds and incremental annotation. We share motivation with these projects, aiming to create an extensible augmentation language and runtime for static math texts~\cite{ref:head2022math}.

Why focus on augmentation in markup editors rather than other sorts of document editors? Markup, and in particular TeX, is used pervasively within the sciences and academia. It is a preferred tool for disseminating and archiving mathematical ideas~\cite{ref:misfeldt2011computers}.
One study suggests writers can enter notation-dense passages more efficiently with TeX than with structured editors~\cite{ref:knauff2014efficiency}. TeX is used widely enough that WYSIWYG editors like Word have incorporated it as a language for formula input~\cite{ref:matthews2019craft}. We see the development of effective markup-based augmentation tools as a natural springboard for efforts to develop better augmentation tools generally.

\paragraph{Structured editors}

WYSIWYG document editors like Word~\cite{tool:wordformulaeditor} sometimes provide structured formula editors. These editors can be used to augment formulas by selecting labels from menus, or by applying their tools for formatting text. Toolkits like MathType make such functionality available as a plugin to other editing applications~\cite{ref:topping1999using}. One advantage of these tools is that their WYSIWYG design makes augmentation affordances easier to discover.

\paragraph{Vector graphics editors}

Formulas can be augmented using vector-based graphics editing software; Head et al.'s study describes Google Slides~\cite{tool:googleslides}, Inkscape~\cite{tool:inkscape}, Mathcha~\cite{tool:mathcha}, and PowerPoint~\cite{tool:powerpoint} as several tools that authors are already using. Some of these tools require authors to render formulas outside of the environment (e.g., with CodeCogs~\cite{tool:codecogs}) and import the render as a bitmap or vector graphics into the editor. These tools are often both familiar to authors and flexible---authors can add augmentations using the full complement of text formatting and shapes the tools provide.  FFL and vector graphics editors occupy two complementary areas of the augmentation design space, with FFL focusing on supporting typesetting experience and transferable styles, and vector graphics editors offering flexible augmentation through direct manipulation.

\paragraph{Sketch and gesture}
Formulas can also be written and augmented as sketches~\cite{ref:laviola2006mathpad2,ref:zeleznik2010hands,ref:leitner2010nice,ref:saquib2021constructing}. In sketching tools, augmentations are naturally supported when authors are given the ability to change ink color and draw free-form shapes. Some sketching tools support unique augmentations, like linking expressions to sketched physical objects~\cite{ref:laviola2006mathpad2,ref:saquib2021constructing}, or manipulating expressions with gestures~\cite{ref:zeleznik2010hands,ref:weitnauer2016graspable,ref:mendes2014structure}. Some of these affordances could be adapted as advanced augmentations for languages like FFL in the future.

\paragraph{Automation}
As text understanding techniques improve, it may be possible to automatically augment notation. Myriad projects have explored the ability to detect the positions of symbols~\cite{ref:head2021augmenting} and parse formulas ~\cite{ref:anthony2005evaluation,ref:ma2021latexify} from arbitrary input documents. Should it become possible to reliably detect expressions and their meaning automatically, augmentations could be added to documents with reduced input from authors.

\subsection{Tools for augmenting texts}
Our work draws inspiration from HCI research that develops powerful text authoring affordances generally.

\paragraph{Repetitive text editing}
One challenge in editing longer texts is making repetitive edits when revising repeated phrases and ideas. HCI research has proposed numerous techniques to do so, including linked editing~\cite{ref:toomim2004managing}, detection and propagation of edits~\cite{ref:miltner2019fly,ref:ni2021recode}, and editable macros~\cite{ref:han2020textlets}. FFL's approach is to allow authors to use CSS-style selectors to indicate which expressions to augment. These selectors allow authors to apply and edit augmentations for many related expressions at once.

\paragraph{Diagrams}
One of the facilities of FFL is to support the creation of simple formula diagrams where descriptive labels are linked to expressions. Researchers have developed powerful domain-specific languages supporting for diagramming like Penrose~\cite{ref:ye2020penrose} and Bluefish~\cite{ref:pollock2022bluefish}. In comparison to these prior toolkits, the aim of FFL's labeling system is to support ease and conciseness in supporting a common, simple sort of labeling, among other augmentations.

\paragraph{Live feedback}
FFL supports third-level or ``edit-triggered'' liveness, according to Tanimoto's taxonomy of liveness~\cite{ref:tanimoto1990viva}. Liveness has been a central feature of dozens of research systems~\cite{ref:rein2018exploratory}. Its use in LaTeX tooling (e.g.,~\cite{ref:gobert2022ilatex,ref:dragicevic2011gliimpse}) may arise from the fact that LaTeX documents require time-consuming compilation to view the effect of a change. FFL incorporates liveness to equip authors with more rapid feedback as they are experimenting with augmentations.
