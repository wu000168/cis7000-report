\section{Discussion}

Our study showed greater speed, ease, and readability of markup code with FFL for the second pair of tasks, which were complex editing tasks. Evidence from the study suggests FFL reduces viscosity, hard mental operations, and error proneness, while providing affordances promoting closeness of mapping and progressive evaluation. These findings suggest the promise of the ideas behind FFL, namely the separation of formula and augmentation markup, live feedback, and its approach to syntax. In this section, we examine the generalizability of the findings and opportunities for advancing the research agenda of which FFL is a part.

\subsection{Limitations}

The generalizability of our findings is necessarily limited to tasks and the sample of participants we studied. When interpreting the results, it is useful to take stock of how authors of web-based math documents would differ from participants in the study.

First, \zed{we anticipate that real-world authors would have greater motivation to use the tools. If an author chose to use FFL, it would reflect a desire to make notation more approachable. We expect a real-world author might therefore experiment more ambitiously with the tools compared to study participants who may not have had prior experience explaining formulas in their writing.}

Second, they would likely be familiar with the formula markup, having written it themselves: for both the LaTeX and FFL conditions, this would likely lead to faster task completion times.

Third, users ``in the wild'' would not have the luxury of having the tools demonstrated over a 10-minute tutorial, and therefore may have more difficulty in a walk-up-and-use experience.

And finally, their FFL markup would have likely gotten longer if they were augmenting a full-length document.
\zed{Our lab study only assigned single-formula tasks because it made it possible for us to select pairs of real-world augmentations where each member of the pair was of approximately equal complexity. Some tasks did require making cross-cutting changes.}
That said, participants in our study did not get a chance to encounter complexities that might arise with longer style specifications, and we did not observe all the difficulties to be seen with scoping augmentation.

Of these limitations, the fourth and fifth are indicators that our lab study reveals only a subset of challenges using FFL; the remaining limitations suggest that task performance could improve for FFL, or both FFL and LaTeX, in more realistic settings. Challenges to using FFL should be further documented by refining the FFL toolkit and evaluating its use in real authoring settings.

\subsection{Future work}

A first line of future research should address opportunities in extending FFL, some already revealed in the study (Section~\ref{sec:shortcomings}).

\paragraph{Scoping} Authors necessarily wish to restrict augmentations to particular expressions, formulas, and passages. While the FFL language provides such capabilities, these were either not discovered or used ineffectively by participants. A future solution could be to let authors specify local ``scopes'' of application in the document markup (e.g., labeling individual passages or formulas) in order to refer to them in selectors.

\paragraph{Resilient expression matching} FFL's current approach to matching token sequences leads to some brittleness in matching expressions that are rendered the same way, but have different LaTeX markup (e.g., in the current implementation, \texttt{\$a\_0\textasciicircum{}1\$} matches ``\texttt{\$a\_0\textasciicircum{}1\$}'' but not ``\texttt{\$a\textasciicircum{}1\_0\$},'' even though they are rendered identically). This was largely not a problem for participants in the study, though we find this undesirable in our own use. FFL provides some flexibility to address cases like these, but we believe a more robust implementation of FFL may benefit from matching patterns with abstract syntax trees, rather than concrete token sequences.

\paragraph{Further improvements} 
As noted in Section~\ref{sec:shortcomings}, FFL should be extended with the ability to apply one label to multiple expressions, more precisely adjust the positions of labels, and better recognize and recover from syntax errors.

\vspace{2.5ex}

This research also points the way to follow-up research on math augmentation that extends into new sorts of tooling.

\paragraph{Direct augmentation}
Some participants desired assistance in writing selectors, understanding selections, and expressing styles. They proposed the ability to directly select them, highlight rendered expressions that are matched by selectors, and generate styles (Section~\ref{sec:shortcomings}). We see FFL as a stepping stone to interactive authoring tools involving direct augmentation like those described by participants, where FFL is used as a substrate, similarly to how backend visualization grammars like Vega-Lite~\cite{ref:satyanarayan2017vegalite} enables visualization exploration interfaces like Voyager~\cite{ref:wongsuphasawat2015voyager}. 

\paragraph{Animated formulas}
FFL was designed to augment static texts, like blog articles or online textbooks. What would an augmentation language look like for dynamic presentations of notation, like animations on the popular \emph{3Blue1Brown}~\cite{ref:3Blue1Brown} YouTube channel for explaining math, where formulas are built up step-by-step and annotated gradually with color and labels? We see FFL as a starting point for developing grammars of animated notation. However, new primitives would have to be designed, as they have in other areas with generalized visualization annotation DSLs for animation~\cite{ref:ge2020canis}.

\paragraph{Making texts interactive}
One pattern of augmentation is creating interactive formulas, where readers can tinker with the values of expressions and see how it influences downstream computations in the formula~\cite{ref:head2022math}.
Prior tools like Idyll~\cite{ref:conlen2018idyll}, \emph{Tangle.js}~\cite{tool:victortangle}, and Potluck~\cite{ref:litt2022potluck} envision the creation of parametric documents where values update reactively as users interact with controls. Extensions to FFL could unify such affordances with its syntax, perhaps even taking advantage of the computation a formula represents to automatically map values in one part of a formula to values elsewhere.

\paragraph{Accessibility}
Augmentations specified in a language like FFL encode additional meaning about a formula, such as what symbols make up meaningful expressions, and what those expressions mean. This information should ideally be surfaced in a way that is accessible to blind and low-vision readers. FFL could be extended to provide cues to screen readers to read a formula aloud in ways that improve upon the default reading order.